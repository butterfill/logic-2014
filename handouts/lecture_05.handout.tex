% !TEX encoding = UTF-8 Unicode
%!TEX TS-program = xelatex

\documentclass[12pt]{extarticle}
% extarticle is like article but can handle 8pt, 9pt, 10pt, 11pt, 12pt, 14pt, 17pt, and 20pt text

\def \ititle {Origins of Mind}
 
\def \isubtitle {Lecture 08}
 
\def \iauthor {Stephen A. Butterfill}
\def \iemail{s.butterfill@warwick.ac.uk}
\date{}

%for strikethrough
\usepackage[normalem]{ulem}

\input{$HOME/Documents/submissions/preamble_steve_handout}

%logic symbol \leftmodels
\usepackage{MnSymbol}

%\bibpunct{}{}{,}{s}{}{,}  %use superscript TICS style bib
%remove hanging indent for TICS style bib
%TODO doesnt work
\setlength{\bibhang}{0em}
%\setlength{\bibsep}{0.5em}


%itemize bullet should be dash
\renewcommand{\labelitemi}{$-$}

\begin{document}

%\raggedcolumns

\begin{multicols*}{3}

\setlength\footnotesep{1em}


\bibliographystyle{newapa} %apalike

%\maketitle
%\tableofcontents




%--------------- 
%--- start paste


\def \ititle {Logic I}
 
\def \isubtitle {Lecture 05}
 
\begin{center}
 
{\Large
 
\textbf{\ititle}: \isubtitle
 
}
 
 
 
\iemail %
 
\end{center}
 
Readings refer to sections of the course textbook, \emph{Language, Proof and Logic}.
 
 
 
\section{¬, $\bot$}
 
\emph{Reading:} §6.3
 
\begin{center}
\includegraphics[scale=0.3]{img/tt_contradiction.png}
\end{center}
\begin{center}
\includegraphics[scale=0.3]{img/rule_contradiction_intro.png}
\end{center}
\begin{center}
\includegraphics[scale=0.3]{img/rule_contradiction_elim.png}
\end{center}
 
 
\section{→Intro, →Elim}
 
\emph{Reading:} §8.1, §8.2
 
\begin{center}
\includegraphics[scale=0.3]{img/rule_arrow_intro.png}
\end{center}
\begin{center}
\includegraphics[scale=0.3]{img/rule_arrow_elim.png}
\end{center}
 
 
\section{→Intro: An Example}
 
\begin{center}
\includegraphics[scale=0.3]{img/proof_arrow_intro.png}
\end{center}
 
 
\section{∨Intro and ∨Elim}
 
\begin{center}
\includegraphics[scale=0.3]{img/rule_disjunction_intro.png}
\end{center}
\begin{center}
\includegraphics[scale=0.3]{img/rule_disjunction_elim.png}
\end{center}
 
 
\section{∨Elim and Soundness}
 
\emph{Reading:} §5.2, §6.2
 
 
 
\section{∨Elim: An Example}
 
To prove a conclusion from a disjunction, prove it from each disjunct.
 
\begin{center}
\includegraphics[scale=0.3]{img/proof_disjunction_elim.png}
\end{center}
 
 
\section{Not Or}
 
\emph{Reading:} §3.7
 
\begin{center}
\includegraphics[scale=0.3]{img/tt_unit_603.png}
\end{center}
 
 
\section{DeMorgan: ¬(A ∧ B) $\leftmodels\models$ ¬A ∨ ¬B}
 
\emph{Reading:} §3.6, §4.2
 
`$\leftmodels\models$' means `is logically equivalent to', so for now `has the same truth table as'.
 
A $\leftmodels\models$ ¬¬A
 
¬(A ∧ B) $\leftmodels\models$ (¬A ∨ ¬B)
 
¬(A ∨ B) $\leftmodels\models$ (¬A ∧ ¬B)
 
A → B $\leftmodels\models$ ¬A ∨ B
 
¬(A → B) $\leftmodels\models$ ¬(¬A ∨ B) $\leftmodels\models$ A ∧ ¬B
 


\section{Ambiguity}
 
Rule 1: a NP followed by a VP is a S
 
Rule 2: a Vt followed by a NP is a VP
 
Rule 3: a NP followed by a PP is a S
 
Rule 4: A Vt followed by a NP then a PP is a VP
 
Two derivations of Groucho Marx’ claim, ‘I shot an elephant in my pyjamas':
 
\begin{center}
\includegraphics[scale=0.3]{img/groucho1.png}
\end{center}
\begin{center}
\includegraphics[scale=0.3]{img/groucho2.png}
\end{center}
\vfill


%--- end paste
%--------------- 
 


\end{multicols*}

\end{document}