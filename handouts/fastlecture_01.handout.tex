%!TEX TS-program = xelatex
%!TEX encoding = UTF-8 Unicode

\documentclass[12pt]{extarticle}
% extarticle is like article but can handle 8pt, 9pt, 10pt, 11pt, 12pt, 14pt, 17pt, and 20pt text

\def \ititle {Origins of Mind}
 
\def \isubtitle {Lecture 08}
 
\def \iauthor {Stephen A. Butterfill}
\def \iemail{s.butterfill@warwick.ac.uk}
\date{}

%for strikethrough
\usepackage[normalem]{ulem}

\input{$HOME/Documents/submissions/preamble_steve_handout}

%\bibpunct{}{}{,}{s}{}{,}  %use superscript TICS style bib
%remove hanging indent for TICS style bib
%TODO doesnt work
\setlength{\bibhang}{0em}
%\setlength{\bibsep}{0.5em}


%itemize bullet should be dash
\renewcommand{\labelitemi}{$-$}

\begin{document}

%\raggedcolumns

\begin{multicols*}{3}

\setlength\footnotesep{1em}


\bibliographystyle{newapa} %apalike

%\maketitle
%\tableofcontents




%--------------- 
%--- start paste

\def \ititle {Logic I}
 
\def \isubtitle {Fast Lecture 01}
 
\begin{center}
 
{\Large
 
\textbf{\ititle}: \isubtitle
 
}
 
 
 
\iemail %
 
\end{center}
 
Readings refer to sections of the course textbook, \emph{Language, Proof and Logic}.
 
 
 
\section{Terminology}
 
\begin{center}
\includegraphics[scale=0.3]{img/name_predicate_sentence.png}
\end{center}
 
 
\section{Logically Valid Arguments}
 
\emph{Reading:} §2.1
 
An argument is \emph{logically valid} just if there’s no possible situation in which the premises are true and the conclusion false
 
A \emph{connective} joins one or more sentences to make a new sentence. E.g. ‘because’, ‘¬’. The sentences joined by a connective are called \emph{constituent sentences}.
 
E.g. in ‘P $\lor{}$ Q’,
 
\begin{quote}
 
$\lor{}$ is the connective
 
P, Q are the constituent sentences
 
\end{quote}
 
\begin{center}
\includegraphics[scale=0.3]{img/terminology_more.png}
\end{center}
 
 
\section{Sentence Letters}
 
\begin{center}
\includegraphics[scale=0.3]{img/sentence_letters.png}
\end{center}
 
 
 
\section{Logic-Ex}
 
There are logic exercises associated with each lecture. After each lecture (or before, if you prefer), you should complete the associated exercises.
 
You can find links to the exercises for each lecture at: \url{http://logic-1.butterfill.com}
 
To complete the exercises you need to register at \url{http://logic-ex.butterfill.com} (If you don’t want to do this, you can complete the alternative textbook exercises on paper. These are also specified for each lecture at \url{http://logic-1.butterfill.com}).
 
Seminars will discuss exercises associated with the previous week’s lectures. As your seminar tutor will track your progress and mark your exercises, you should be sure to \textbf{complete the exercises by 2pm on the day before your seminar}.
 

 
\section{Counterexamples}
 
\emph{Reading:} §2.5
 
A \emph{counterexample} to an argument is a possible situation in which its premises are T and its conclusion F.
 
There are no counterexamples to a logically valid argument.
 
If an argument is not valid, then there is a counterexample to it.
 
To show that an argument is not logically valid, we specify a counterexample to it.
 
 
 
\section{Identity}
 
\emph{Reading:} §2.2
 
Principle: If b=c then whatever is true of b is also true of c.
 
Principle: a=a is never false
 
\begin{center}
\includegraphics[scale=0.3]{img/arg_identity.png}
\end{center}
 
 
\section{Truth Tables}
 
\emph{Reading:} §3.1, §3.2, §3.3
 
Rough guide:
 
`$\land{}$' means and
 
`$\lor{}$' means or
 
`$\lnot{}$' means not
 
\begin{center}
\includegraphics[scale=0.3]{img/truth_table_or_and.png}
\end{center}
\begin{center}
\includegraphics[scale=0.3]{img/truth_table_not.png}
\end{center}
 
 
\section{Complex Truth Tables}
 
\emph{Reading:} §3.3, §3.5
 
\begin{center}
\includegraphics[scale=0.3]{img/how_to_write_truth_tables.png}
\end{center}
\begin{minipage}{\columnwidth}
 
Complex truth table example:
 
\begin{center}
\includegraphics[scale=0.3]{img/tt_p_and_q_or_r.png}
\end{center}
\end{minipage}
 
 
 
\section{Logical Validity and Truth Tables}
 
\emph{Reading:} §4.3
 

 
\begin{minipage}{\columnwidth}
 
To establish that an argument is valid:
 
\begin{enumerate}
 
\item Create truth tables for each premise and the conclusion.
 
\item Check whether there is a row of the truth table where all premises are true and the conclusion is false.
 
\item If not, the argument is valid.
 
\end{enumerate}
 
\end{minipage}
 
 
 
\section{Tautologies and Contradictions}
 
\emph{Reading:} §4.1, §4.2
 
\begin{center}
\includegraphics[scale=0.3]{img/unit_160_argument3.png}
\end{center}
\begin{center}
\includegraphics[scale=0.3]{img/unit_160_argument3b.png}
\includegraphics[scale=0.3]{img/unit_160_argument4.png}
\end{center}
P $\lor{}$ ¬P is a \emph{logical truth}
 
logical truth defined p. 568
 
P $\lor{}$ ¬P is a \emph{contradiction}
 
contradiction defined p. 564
% 
%\vfill
%\begin{minipage}{\columnwidth}
%\section{Exercises}
%These exercises will be discussed in seminars the week after this lecture.
%The numbers below refer to the numbered exercises in the course textbook, e.g. `1.1' refers to exercise 1.1. on page 39 of the second edition of \emph{Language, Proof and Logic}.
% 
%\begin{quote}
%2.8, 2.10, 2.12, 2.21
% 
%3.1, 3.3
% 
%3.5, 3.7
% 
%3.14, 3.15
% 
%4.1, 4.2
% 
%4.12--4.16
% 
%\end{quote}
%\end{minipage}
%
%%--- end paste
%%--------------- 
% 

\end{multicols*}

\end{document}